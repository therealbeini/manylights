%% LaTeX2e class for student theses
%% sections/abstract_de.tex
%% 
%% Karlsruhe Institute of Technology
%% Institute for Program Structures and Data Organization
%% Chair for Software Design and Quality (SDQ)
%%
%% Dr.-Ing. Erik Burger
%% burger@kit.edu
%%
%% Version 1.3.3, 2018-04-17

\Abstract

Wenn wir Szenen rendern, in denen wir hohe Anforderungen auf die Qualität des Bilds setzen, ist path tracing das beliebteste Werkzeug in den letzten Jahren gewesen. Verglichen mit Rasterisierung liefert path tracing eine höhere Bildqualität mit einer längeren Renderzeit. Das ist der Hauptgrund, warum path tracing der bevorzugte Algorithmus ist, wenn man Bilder oder Videos im Voraus erstellt. Allerdings sind die Renderzeiten von von animierten Filmen wie der in 2014 veröffentlichte Film \textit{Big Hero 6}, durch extrem komplexe Szenen explodirt. Diesney hat Statistiken veröffentlicht, die zeigen, dass der Film mit 1,1 Millionen Rechenstunden pro Tag, aufgeteilt auf einem 55.000-Kern Computer über vier geographische Orte, gerendert worden ist. Wenn die Renderzeiten so lang sind, dann sind sie wichtig, auch wenn das System den Film nicht in Echtzeit rendern muss. \textit{Big Hero 6} spielt in einer Stadt namens San Fransokyo. Sie umfasst 83.000 Gebäude, 215.000 Straßenlaternen und 100.000 Fahrzeuge. Nachts können all diese Objekte Lichtquellen sein, was zu einer extrem hohen Zahl von Lichtern führt. Wenn wir mit path tracing rendern, müssen wir bestimmte Punkte der Szene mit Lichtquellen beleuchten.  Mit dieser großen Zahl von Lichtern, ergibt es natürlich keinen Sinn einen einzelnen Punkt mit jeder einzelnen Lichtquelle der Szene zu beleuchten. Ein zufälliges Licht für die Beleuchtung zu nehmen, ist zwar eine Möglichkeit, aber die Bildqualität wird stark darunter leiden.

In dieser Arbeit führen wir eine Beschleunigungsstruktur ein, die es uns erlaubt, schneller Szenen mit eine großen Zahl von Lichtquellen zu rendern. Statt ein zufälliges Licht der Szene zu nehmen, werden wir versuchen, öfters Lichtquellen abzutasten, die einen hohen Einfluss auf den Punkt haben, den wir beleuchten wollen. Somit konvergiert unser Algorithmus schneller und die Bildqualität wird bei gleichen Renderzeit besser sein als mit konventionellen Methoden, die Lichtquellen aussuchen.