%% LaTeX2e class for student theses
%% sections/abstract_de.tex
%% 
%% Karlsruhe Institute of Technology
%% Institute for Program Structures and Data Organization
%% Chair for Software Design and Quality (SDQ)
%%
%% Dr.-Ing. Erik Burger
%% burger@kit.edu
%%
%% Version 1.3.3, 2018-04-17

\Abstract

Wenn wir Szenen rendern, in denen wir hohe Anforderungen auf die Qualität des Bilds setzen, ist path tracing das beliebteste Werkzeug in den letzten Jahren gewesen. Verglichen mit Rasterisierung liefert path tracing eine höhere Bildqualität mit einer längeren Renderzeit. Das ist der Hauptgrund, warum path tracing der bevorzugte Algorithmus ist, wenn man Bilder oder Videos im Voraus erstellt. Neuerdings ist Echtzeit ray tracing ein großes Thema in der Computergrafik geworden. Die Renderzeiten so kurz wie möglich zu halten, ist in beiden Fällen wünschenswert.

In dieser Arbeit führen wir eine Beschleunigungsstruktur ein, die es uns erlaubt, schneller Szenen mit eine großen Zahl von Lichtquellen zu rendern. Statt ein zufälliges Licht der Szene zu nehmen, werden wir versuchen, öfters Lichtquellen abzutasten, die einen hohen Einfluss auf den Punkt haben, den wir beleuchten wollen. Somit konvergiert unser Algorithmus schneller und die Bildqualität wird bei gleichen Renderzeit besser sein als mit konventionellen Methoden, die Lichtquellen aussuchen.