%% LaTeX2e class for student theses
%% sections/content.tex
%% 
%% Karlsruhe Institute of Technology
%% Institute for Program Structures and Data Organization
%% Chair for Software Design and Quality (SDQ)
%%
%% Dr.-Ing. Erik Burger
%% burger@kit.edu
%%
%% Version 1.3.3, 2018-04-17

\chapter{Introduction}
\label{ch:Introduction}

\section{Problem/Motivation}
\label{sec:Introduction:Motivation}

Path tracing is one of the most important rendering techniques when creating highly realistic pictures. It allows us to render the scene much closer to reality when compared to typical scanline rendering methods at the cost of more computations. In situations where the images can be rendered ahead of time, such as for visual effects or films, we can take advantage of the better results of ray tracing. Recently, there have been talks about real time ray tracing. NVIDIA claims that real-time ray tracing with a single GPU in games and other graphic applications was possible. Keeping the rendering times as short as possible is not exclusive for real-time ray tracing. Animated films like the in 2014 released \textit{Big Hero 6}, have exploded with more and more complicated scenes. Disney released statistics showing the movie was rendered with 1.1 million computational hours per day, distributed on a 55.000-core computer across four geographic locations. If the rendering times are so long, they are important even when the system does not need to provide the rendering of the movie in real-time. \textit{Big Hero 6} plays in a city called San Fransokyo. It contains around 83.000 buildings, 215.000 streetlights and 100.000 vehicles. At nighttime, all these objects can be light sources, which leads to a gigantic number of emitters. When rendering with path tracing, we have to light certain points of the scene with light sources. With this huge number of lights, it is clearly not practical to calculate the lighting of each point for every single light in the scene. Also, choosing random light sources will definitely not achieve the image quality we want for a pleasant user experience in a reasonable time. \cite{BH6,NVIDIA}

There are other sampling approaches that try to limit the time required to render these scenes with a big amount of lights. For instance, we could say that the probability of a point of the scene sampling a certain light is only dependent on the emission power of said light. We would make a distribution that only takes into account the emission power of the lights. To light a specific point we would then sample a single light according to the distribution function we built earlier. Obviously there are a lot of problems with this approach. An area light source or a spotlight could be facing towards a completely different direction and may not have any effect on the point. Or the light source could be potentially too far away to have a noticeable effect on the point. This sampling technique asserts a faster sampling speed but can lead to very noisy images that we are trying to avoid.

In this bachelor's thesis we discuss a light sampling technique that improves the rendering speed in scenes with many emitters while still maintaining an comparable quality.

\section{Content}
\label{sec:Introduction:Content}

We will introduce the light bounding volume hierarchy as an acceleration data structure to render scenes with complex illumination. In chapter \ref{ch:preliminaries}, we will talk about some concepts of probability theory and path tracing that are required for the algorithm discussed in this thesis. Then, in chapter \ref{ch:alg}, we will give an in-depth introduction of the algorithm. In chapter \ref{ch:Evaluation}, we will evaluate the algorithm. In chapter \ref{ch:related}, we will reference related work and in chapter \ref{ch:Conclusion} we will come to a conclusion.