%% LaTeX2e class for student theses
%% sections/content.tex
%% 
%% Karlsruhe Institute of Technology
%% Institute for Program Structures and Data Organization
%% Chair for Software Design and Quality (SDQ)
%%
%% Dr.-Ing. Erik Burger
%% burger@kit.edu
%%
%% Version 1.3.3, 2018-04-17

\chapter{Introduction}
\label{ch:Introduction}

\section{Problem/Motivation}
\label{sec:Introduction:Motivation}

Ray tracing is one of the most important rendering techniques when the aim is to create highly realistic pictures. It allows us to render the scene much closer to reality compared to typical scanline rendering methods at the cost of more computations. In situations where the images can be rendered ahead of time, such as for visual effects or films, we can take advantage of the better results of ray tracing. Then again, ray tracing is not useful for real-time applications like video games where the rendering speed is critical. But even when it comes to ray tracing, we cannot completely ignore the rendering time. Too long rendering times are becoming a problem in scenes with many lights. For instance a scene of big city with skyscrapers at night could have hundreds or thousands of lights that could potentially all affect a single point in the scene. Typical light sampling methods would be too slow to deal with these situations, since we cannot calculate the effect of every single light on the sampled point.

There are sampling approaches that try to limit the time required to render these scenes with a big amount of lights. For instance, we could say that the probability of a point of the scene being sampled by a certain light is only dependent on the emission power of said light. We would make a distribution that only takes into account the emission power of the lights. To light a point we would then pick out a single light with a random number generator and sample the point with that light. Obviously there are a lot of problems with this approach. An area light source or a spotlight could be facing towards a completely different direction and not have any effect on the point. Or the light could be potentially too far away to have a noticeable effect on the point. This sampling technique asserts a fast sampling speed but can lead to very noisy images that we try to avoid.

For this bachelor thesis I have implemented a light sampling technique that optimizes the rendering speed without making the rendered image too noisy.

\section{Content}
\label{sec:Introduction:Content}