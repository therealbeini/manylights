%% LaTeX2e class for student theses
%% sections/evaluation.tex
%% 
%% Karlsruhe Institute of Technology
%% Institute for Program Structures and Data Organization
%% Chair for Software Design and Quality (SDQ)
%%
%% Dr.-Ing. Erik Burger
%% burger@kit.edu
%%
%% Version 1.3.3, 2018-04-17

\chapter{Related Work and Future Work}
\label{ch:related}

\section{Lightcuts}

Lightcuts \Cite{LCT} tried to deal with the same problem as we did very early on in 2005. In scenes with complex illumination with many light sources, Lightcuts tries to approximate the direct lighting by substituting a cluster of lights by a single representational light source. So, in situations, where we would want to illuminate an intersection point with every light source in the scene, we can instead illuminate the point with a smaller subset of light sources. That leads to a model that scales much better with many light sources in the scene.

The basic algorithm follows a simple concept that can be divided into certain steps:

\begin{itemize}
	\item Convert illumination to point lights
	\item Build light tree
	\item For each eye ray, choose a cuts to approximate the illumination
\end{itemize}

We will break down the single steps for the reader. First, the complex illumination of the scene is substituted by point lights with a similar lighting. The reason for that is that the algorithm required point lights to run. Next, a binary tree that includes every emitter of the scene is built, with the interior nodes representing light clusters and the leafs representing single emitters. When rendering, for each eye ray, the algorithm calculates a cut that minimizes the number of lights sampled, while guaranteeing a certain quality threshold. A cut of the binary tree is a substitution of the total number of emitters in the scene. Either, the cut includes the actual light source or a cluster as a substitution of said light source. Therefore, when we light a intersection point with a cut, we limit the number of sampled lights.

To find out a suitable cut for an eye ray, the algorithm will start with the cluster that substitutes the whole lighting of the scene with a certain light. Gradually, the algorithm will increase the number of representative lights, as long as the quality of the lighting would increase over a certain threshold. If none of these substitutions will yield us a noticeable increase in quality, we will stop the algorithm and work with the cut calculated.

The paper further introduces reconstruction cuts, that calculates the color of certain positions sparsely over the image and then interpolates their illumination to shade the rest of the image.

While lightcuts are a interesting approach to dealing with scenes with complex illumination, while path tracing, we would have to calculate much more cuts to illuminate the scene. Also, due to it's approximating nature, the rendered image will never converge to the ideal image.

\section{Importance Sampling of Many Lights with Adaptive Tree Splitting}

The contribution of Conty and Kulla \Cite{MLA,MLS} was basically the source of the idea for this algorithm. Many of the things we have explained has been implementated the same way or similarly in their works. We have marked the spots in this thesis where we drastically changed our algorithm compared to their ideas. In the scenes we have tested our version of the algorithm, the images appear less noisy. Sadly, our splitting algorithm could not meet the qualities of the quality they have described in their presentation.

\section{Future Work}

As we have mentioned in the previous section, our splitting algorithms does not have the desirable quality. Obviously, that is the first thing we want to improve. Being able to archive a higher quality with a shorter rendering time is just what this thesis is about.

Next, we would like to be able to use our algorithm in scenes that are not only lighted by point lights, spotlights and area light sources. For instance, we would like to implement distant light sources like the illumination of the sun or projection light sources that model the lighting of a slide projector.

Finally, at some points we have tried optimized our algorithm, for instance by creating a more compact data structure for a faster tree traversal. On other points, we have favored keeping the code in a more readable state instead of optimizing it for development reasons. When we are sure that we will not change too much in our implementation anymore, it would clearly be desirable to optimize the code for a faster rendering time.