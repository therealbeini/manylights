%% LaTeX2e class for student theses
%% sections/abstract_en.tex
%% 
%% Karlsruhe Institute of Technology
%% Institute for Program Structures and Data Organization
%% Chair for Software Design and Quality (SDQ)
%%
%% Dr.-Ing. Erik Burger
%% burger@kit.edu
%%
%% Version 1.3.3, 2018-04-17

\Abstract

When rendering complex scenes with a high requirement for image quality, path tracing has been the most popular tool in the last few years. Compared to scanline techniques like rasterization, path tracing offers a higher image quality with the trade-off of longer rendering times. That is the main reason why path tracing is the preferred algorithm when creating pictures or videos in beforehand. Recently, real-time ray tracing has also become a big subject in computer graphics. Keeping the rendering times as short as possible is desirable in both cases.

This work discusses an acceleration structure that allows for faster rendering of scenes with a large number of lights. Instead of choosing a random light source in the scene, we will try to sample lights that have a high contribution to the point to be lighted more often. That way, our algorithm converges faster and the image quality with similar rendering times will be better than using conventional light sampling strategies.